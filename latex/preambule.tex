%%%%Prambule du fichier principal: les lignes suivantes, avant \begin{document} et le corps du texte:
%\documentclass[spanish,twoside,a4paper]{book}



%R\UTF{00E9}soud le probl\UTF{00E8}me de registre (No room for...)
\usepackage{etex}
%%%%%%%%%%r\UTF{00E9}soud le probSleme de registre%%%%%%%

\reserveinserts{28}
\usepackage[T1]{fontenc} 
\usepackage{fontspec}
\setmainfont{Junicode}
%\usepackage{fbb}
\usepackage[latin, english, french, spanish]{babel}
\usepackage{lipsum}

%Diffrents types de guillemets selon la tradition du pays
\usepackage{csquotes}
%Diffrents types de guillemets selon la tradition du pays

%sets the footnote marker flush with,  but just inside the margin from, the text of the footnote; This option forces footnotes to the bottom of the page
%\usepackage[bottom,flushmargin]{footmisc}
%sets the footnote marker flush with,  but just inside the margin from, the text of the footnote; This option forces footnotes to the bottom of the page


\newcommand{\codeword}[1]{%
\textcolor{blue}{#1}%
}


%permet de supprimer l'entte/pied de page pour les pages vides aprs un \cleardoublepage
\usepackage{emptypage}
%permet de supprimer l'entte/pied de page pour les pages vides aprs un \cleardoublepage



\usepackage{url}

%FAIRE DES BOITES
\usepackage{mdframed}
%FAIRE DES BOITES


%contrle de la mise en page gnrale
\usepackage{layout}
%contrle de la mise en page gnrale



\usepackage{changepage}
\newenvironment{note_quote}
    {\begin{adjustwidth}{.03\columnwidth}{.03\columnwidth}}
    {\end{adjustwidth}}
    



%Bonne hyphenation des urls
\makeatletter
\g@addto@macro{\UrlBreaks}{\UrlOrds}
\makeatother
%Bonne hyphenation des urls


\usepackage[top=2cm, bottom=1.5cm, left=2cm,right=2cm, heightrounded, marginparwidth=2.5cm, marginparsep=0.8cm]{geometry}


%stemma
\usepackage{tikz}
\usetikzlibrary{shapes}
%stemma


%ESPACE ENTRE LES LIGNES
%\DisemulatePackage{setspace}%n\UTF{00E9}cessaire pour que \UTF{00E7}a marche avec memoir
\usepackage{setspace}
\onehalfspacing
%ESPACE ENTRE LES LIGNES


%%%%%%%%%%%%%%%%%%%%%%%%%%%STRUCTURE%%%%%%%%%%%%%%%%%%%%%%%%%%%%%%%%%%%%
%Je sais pas  quoi a sert
\usepackage{etoolbox}
%Je sais pas  quoi a sert
\usepackage{ifthen}

%%% HEADER FOOTER
%replacement?partial or total?for the LATE X macros related with sections?namely titles, headers and contents
\usepackage[pagestyles,extramarks]{titlesec}
%replacement?partial or total?for the LATE X macros related with sections?namely titles, headers and contents
%\settitlemarks*{section,subsection,chapter}
%parties principales du mmoire
%\newpagestyle{commentaire}{
%\setfoot{}{\thepage}{}
%\sethead[ \ifthechapter{}{\large\quad\textsc{\firstextramarks{chapter}\chaptertitle}}][][]{}{}{\large\textit{\sectiontitle}\quad}
%\headrule}
%parties principales du mmoire

\newpagestyle{edition_vis_a_vis}{
\setfoot{}{\thepage}{}
\sethead[Chapitre \thechapter][Traduction][Version A]{Version B}{Traduction}{Chapitre \thechapter}
\headrule}


\newpagestyle{edition_glose}{
\sethead{Version B}{Glose}{Chapitre \thechapter}
\setfoot[][][]{}{\thepage}{}
\headrule}




%\assignpagestyle{\chapter}{toc}
%\assignpagestyle{\part}{tic}





%corps de texte de l'dition
\newpagestyle{texte2}{
\setfoot{}{\thepage}{}
\sethead[\quad\large\textit{\subsectiontitle}][][]{}{}{\quad\large \quad \Large\textsc{Livre i} -- \large\sectiontitle\quad}
\headrule}
%corps de texte de l'dition



\usepackage{tocloft}
\renewcommand\cftchapaftersnum{.}% adds dot after chapter title in ToC
\renewcommand\cftchapdotsep{\cftdotsep}
%toc dans la toc
\usepackage{tocbibind}
%toc dans la toc

%profondeur de la numrotation structurelle
\setcounter{secnumdepth}{0}
%profondeur de la numrotation structurelle

%RESET NUMERO CHAPITRE
\usepackage{remreset}%reset des num\UTF{00E9}ros de notes/chapitres
\makeatletter
\@removefromreset{footnote}{}
\makeatother
%RESET NUMERO CHAPITRE


%%%%%Red\UTF{00E9}finition des diff\UTF{00E9}rents titres de section%%%%%
\titleformat{\chapter}[display]
    {\normalfont\huge\bfseries}{\chaptertitlename\ \thechapter}{10pt}{\huge}
\titlespacing*{\chapter}{-20pt}{-50pt}{10pt}%% > haut de page. La deuxi\UTF{00E8}me fenetre de r\UTF{00E9}glage fait la hauteur avant le titre, la premi\UTF{00E8}re gauche/droite, la derni\UTF{00E8}re la hauteur apr\UTF{00E8}s le titre. 
\titleformat{\section}
  {\normalfont\LARGE\bfseries}{\thesection}{1em}{}
\titlespacing*{\section}{-20pt}{40pt}{10pt}
%\interfootnotelinepenalty=10000
\titleformat{\subsection}
  {\normalfont\scshape\Large\bfseries}{\thesubsection}{1em}{}
  \titlespacing*{\subsection}{10pt}{30pt}{10pt}
\titleformat{\subsubsection}
  {\sffamily\large\itshape%\filcenter
  }{\thesubsubsection}{1em}{}
  \titlespacing*{\subsubsection}{5pt}{20pt}{30pt}
%%%%%Red\UTF{00E9}finition des diff\UTF{00E9}rents titres de section%%%%%




%Grosseur d'indentation
\parindent=1cm
%Grosseur d'indentation

%%%%%%%%%%%%%%%%%%%%%%%%%%%STRUCTURE%%%%%%%%%%%%%%%%%%%%%%%%%%%%%%%%%%%%





% Commencer les chapitres de l'dition  et la biblio en page impaire (ajouter \cleartorightpage dans la template d'impression du titre de chaque chapitre)
\makeatletter
\def\cleartorightpage{\clearpage\if@twoside \ifodd\c@page\else
\hbox{}\thispagestyle{empty}\newpage\fi\fi}
\makeatother
\makeatletter
\def\cleartorightpage{\clearpage\if@twoside \ifodd\c@page\else
\hbox{}\thispagestyle{biblio}\newpage\fi\fi}
\makeatother
\makeatletter
\def\cleartorightpage{\clearpage\if@twoside \ifodd\c@page\else
\hbox{}\thispagestyle{toc}\newpage\fi\fi}
\makeatother
% Commencer les chapitres de l'dition en page impaire



\usepackage[series={A,B,C,D},noend,noeledsec,noledgroup]{reledmac}
\usepackage{reledpar}
\renewcommand{\footfudgefiddle}{68}
\arrangementX[A]{normal}
\arrangementX[B]{normal}
\Xarrangement[B]{paragraph}
\Xarrangement[C]{paragraph}
\Xarrangement[D]{paragraph}
\Xlemmaseparator[C]{~}
\Xlemmaseparator[D]{~}

\onlysideX[A]{L}
\onlysideX[B]{R}
\Xonlyside[B]{R}
\Xonlyside[D]{R}
\Xbeforelemmaseparator[D]{-.4em}
% https://githubmemory.com/repo/maieul/ledmac/issues/870
\setgoalfraction{.70}

\AtBeginDocument{%
\Xmaxhnotes{0.6\textheight}
\maxhnotesX{0.6\textheight}
}


%%%%%%%%%%%%%%%%%%%%%%%%%%%%%%%%%BIBLIO%%%%%%%%%%%%%%%%%%%%%%%%%%%%%%%%%%%%%
\usepackage{xpatch}
\usepackage{filecontents}


% https://github.com/maieul/biblatex-manuscripts-philology/issues/24

\usepackage[
    tools=manuscripts,
    bibstyle=authoryear,
]{biblatex-multiple-dm}
\usepackage[bibstyle=multiple-dm,citestyle=authoryear, isbn=false, doi=true,backend=biber,language=french,url=true,maxbibnames=99]{biblatex}

% https://github.com/maieul/biblatex-manuscripts-philology/issues/24


%Pas de pagebreak avant la biblio
\defbibheading{secbib}[\bibname]{%
 % \section*{#1}%
 % \markboth{#1}{#1}
 }
  %Pas de pagebreak avant la biblio
%\renewcommand*{\mkibid}{\emph}%  mettre avec un style qui utilise ibid (pour rendre ibid italique).
\renewcommand{\mkbibnamelast}[1]{\textsc{#1}}
\DeclareFieldFormat[report]{title}{\textit{{#1}}}
\DeclareFieldFormat[article]{journaltitle}{{\textit{{#1}}}}
  \DeclareFieldFormat[book]{title}{{\textit{#1}}}
    \DeclareFieldFormat[misc]{title}{{\textit{#1}}}
    \DeclareFieldFormat[book]{booktitle}{{\textit{#1}}}
   \DeclareFieldFormat[inbook]{booktitle}{{\textit{#1}}}
  \DeclareFieldFormat[techreport]{title}{{\textit{#1}}}
\DeclareLanguageMapping{francais}{francais-apa}

%change In: par \textit{in}
\DefineBibliographyStrings{french}{
  in = {\textit{in}},
}
\renewbibmacro{in:}{%
  \ifentrytype{article}{}{\printtext{\bibstring{in}\intitlepunct}}}
%change In: par \textit{in}

%ponctuation entre  \UTF{00E9}l\UTF{00E9}ments: virgule
    \renewcommand{\newunitpunct}[0]{, }
%ponctuation entre  \UTF{00E9}l\UTF{00E9}ments: virgule

\makeatletter
\AtEveryCitekey{%
  \ifboolexpr{ test {\iffieldequalstr{entrysubtype}{classical}}
               and not test {\iffieldundef{shorttitle}} }
    {\ifciteseen
       {\blx@ibidreset\clearname{labelname}}
       {\savefield{title}{\cbxtitle}\restorefield{labeltitle}{\cbxtitle}}}
    {}}
\makeatother

% https://tex.stackexchange.com/a/424775
% print url if no doi
\renewbibmacro*{doi+eprint+url}{%
    \printfield{doi}%
    \newunit\newblock%
    \iftoggle{bbx:eprint}{%
        \usebibmacro{eprint}%
    }{}%
    \newunit\newblock%
    \iffieldundef{doi}{%
        \usebibmacro{url+urldate}}%
        {}%
    }

% https://tex.stackexchange.com/a/184878 ajout direction thèse

\newbibmacro*{thesissupervisor}{%
  \ifnameundef{editor}{}{%
    \ifnumgreater{\value{editor}}{1}
      {\bibstring{, codirigée par}}
      {\bibstring{, dirigée par}}
    \printnames{editor}}}

\xpatchbibdriver{thesis}
  {\printfield{type}}
  {\printfield{type}
   {\normalfont \usebibmacro{thesissupervisor}}}
  {\typeout{yep}}
  {\typeout{no}}
  
  
% https://tex.stackexchange.com/a/184878 ajout direction thèse

%https://tex.stackexchange.com/questions/451821/distinguishing-between-a-scientific-editor-and-the-editor-of-a-text-in-biblatex
%comment créer un nouveau champ qui permette de différencier éditeur (ecdotique) et éditeur d'un volume.
\NewBibliographyString{scied}
\NewBibliographyString{scieds}
\NewBibliographyString{byscied}
\DefineBibliographyStrings{french}{
  scied   = {éd\adddot},
  scieds  = {éd\adddot},
  byscied = {éd\adddotspace par},
}


% https://tex.stackexchange.com/a/74298 pour avoir les bibliographies qui se comportent comme des sections
\defbibheading{secbib}[\bibname]{%
  \section*{#1}%
  \markboth{#1}{#1}}


%%%%%%%%%%%%%%%%%%%%%%%%%%%%%%%%%BIBLIO%%%%%%%%%%%%%%%%%%%%%%%%%%%%%%%%%%%%%
%Attention au chemin absolu ici !
\addbibresource{/home/mgl/Bureau/These/Edition/hyperregimiento-de-los-principes/Dedans/XML/corpus/biblio.bib}
\addbibresource{/home/mgl/Bureau/These/Edition/hyperregimiento-de-los-principes/Dedans/XML/corpus/biblio_mss.bib}
\addbibresource{/home/mgl/Bureau/These/Edition/hyperregimiento-de-los-principes/Dedans/XML/corpus/primary_sources.bib}
\title{Bibliographie}
%\date{}%Enlever la date quand on utilise \maketitle
%\usepackage[urlcolor=green]{hyperref}%r\UTF{00E9}f\UTF{00E9}rences internes au document (toc, etc) \UTF{00E0} mettre en dernier
\usepackage{hyperref}




%%%%%%%%%%%%%%%%%%%%%%%%%%%NOTES DE BAS DE PAGE ET APPARAT%%%%%%%%%%%%%%%%%%%%%%%%%%


%notes marginales
\usepackage{marginnote}
%notes marginales









