%%%%Prambule du fichier principal: les lignes suivantes, avant \begin{document} et le corps du texte:
%\documentclass[francais,twoside,a4paper]{book}
%R\UTF{00E9}soud le probl\UTF{00E8}me de registre (No room for...)

%%%%%%%%%%r\UTF{00E9}soud le probleme de registre%%%%%%%
%Utiliser un fichier externe pour la bibliographie%
%\usepackage{subfiles}
%Utiliser un fichier externe pour la bibliographie%
%\input{chemin du prambule}


\usepackage{etex}
\reserveinserts{28}
\usepackage[T1]{fontenc} 
\usepackage[utf8]{inputenc}
\usepackage{fbb}
\usepackage[spanish]{babel}
\usepackage{lipsum}

%Diffrents types de guillemets selon la tradition du pays
\usepackage{csquotes}
%Diffrents types de guillemets selon la tradition du pays



%permet de supprimer l'entte/pied de page pour les pages vides aprs un \cleardoublepage
\usepackage{emptypage}
%permet de supprimer l'entte/pied de page pour les pages vides aprs un \cleardoublepage



\usepackage{lettrine}
\usepackage{url}

%FAIRE DES BOITES
\usepackage{mdframed}
%FAIRE DES BOITES


%contrle de la mise en page gnrale
\usepackage{layout}
%contrle de la mise en page gnrale



%Bonne hyphenation des urls
\makeatletter
\g@addto@macro{\UrlBreaks}{\UrlOrds}
\makeatother
%Bonne hyphenation des urls


%\usepackage[top=2.5cm, bottom=1.5cm, left=3cm,right=2cm, heightrounded, includefoot]{geometry}

\usepackage[top=2.5cm, bottom=1.5cm, left=2.25cm,right=2cm, heightrounded, marginparwidth=2.5cm, marginparsep=0.8cm, includefoot]{geometry}
%%%%%%%%%%%%%%RESOUD PROBLEME EDNOTES NEWGEOMETRY
\makeatletter
    \newcommand*{\newmanyfootgeometry}[1]{%
        \newgeometry{#1}\MFL@columnwidth\columnwidth}
    \makeatother
\raggedbottom
%%%%%%%%%%%%%%RESOUD PROBLEME EDNOTES NEWGEOMETRY
%stemma
\usepackage{tikz}
\usetikzlibrary{shapes}
%stemma


%ESPACE ENTRE LES LIGNES
%\DisemulatePackage{setspace}%n\UTF{00E9}cessaire pour que \UTF{00E7}a marche avec memoir
\usepackage{setspace}
\onehalfspacing
%ESPACE ENTRE LES LIGNES


%%%%%%%%%%%%%%%%%%%%%%%%%%%STRUCTURE%%%%%%%%%%%%%%%%%%%%%%%%%%%%%%%%%%%%
%Je sais pas  quoi a sert
\usepackage{etoolbox}
%Je sais pas  quoi a sert
\usepackage{ifthen}

%%% HEADER FOOTER
%replacement?partial or total?for the LATE X macros related with sections?namely titles, headers and contents
\usepackage[pagestyles,extramarks]{titlesec}
%replacement?partial or total?for the LATE X macros related with sections?namely titles, headers and contents
\settitlemarks*{section,subsection,chapter}
%parties principales du mmoire
\newpagestyle{commentaire}{
\setfoot{}{\thepage}{}
\sethead[ \ifthechapter{}{\large\quad\textsc{\firstextramarks{chapter}\chaptertitle}}][][]{}{}{\large\textit{\sectiontitle}\quad}
\headrule}
%parties principales du mmoire



\assignpagestyle{\chapter}{toc}
\assignpagestyle{\part}{tic}





%corps de texte de l'dition
\newpagestyle{texte2}{
\setfoot{}{\thepage}{}
\sethead[\quad\large\textit{\subsectiontitle}][][]{}{}{\quad\large \quad \Large\textsc{Livre i} -- \large\sectiontitle\quad}
\headrule}
%corps de texte de l'dition



\usepackage{tocloft}
\renewcommand\cftchapaftersnum{.}% adds dot after chapter title in ToC
\renewcommand\cftchapdotsep{\cftdotsep}
%toc dans la toc
\usepackage{tocbibind}
%toc dans la toc

%profondeur de la numrotation structurelle
\setcounter{secnumdepth}{0}
%profondeur de la numrotation structurelle

%RESET NUMERO CHAPITRE
\usepackage{remreset}%reset des num\UTF{00E9}ros de notes/chapitres
\makeatletter
\@removefromreset{footnote}{}
\makeatother
%RESET NUMERO CHAPITRE


%%%%%Red\UTF{00E9}finition des diff\UTF{00E9}rents titres de section%%%%%
\titleformat{\chapter}[display]
    {\normalfont\huge\bfseries}{\chaptertitlename\ \thechapter}{10pt}{\huge}
\titlespacing*{\chapter}{-20pt}{-50pt}{10pt}%% > haut de page. La deuxi\UTF{00E8}me fenetre de r\UTF{00E9}glage fait la hauteur avant le titre, la premi\UTF{00E8}re gauche/droite, la derni\UTF{00E8}re la hauteur apr\UTF{00E8}s le titre. 
\titleformat{\section}
  {\normalfont\LARGE\bfseries}{\thesection}{1em}{}
\titlespacing*{\section}{-20pt}{40pt}{10pt}
\interfootnotelinepenalty=10000
\titleformat{\subsection}
  {\normalfont\scshape\Large\bfseries}{\thesubsection}{1em}{}
  \titlespacing*{\subsection}{10pt}{30pt}{10pt}
\titleformat{\subsubsection}
  {\sffamily\large\itshape%\filcenter
  }{\thesubsubsection}{1em}{}
  \titlespacing*{\subsubsection}{5pt}{20pt}{30pt}
%%%%%Red\UTF{00E9}finition des diff\UTF{00E9}rents titres de section%%%%%




%Grosseur d'indentation
\parindent=1cm
%Grosseur d'indentation

%%%%%%%%%%%%%%%%%%%%%%%%%%%STRUCTURE%%%%%%%%%%%%%%%%%%%%%%%%%%%%%%%%%%%%





% Commencer les chapitres de l'dition  et la biblio en page impaire (ajouter \cleartorightpage dans la template d'impression du titre de chaque chapitre)
\makeatletter
\def\cleartorightpage{\clearpage\if@twoside \ifodd\c@page\else
\hbox{}\thispagestyle{empty}\newpage\fi\fi}
\makeatother
\makeatletter
\def\cleartorightpage{\clearpage\if@twoside \ifodd\c@page\else
\hbox{}\thispagestyle{biblio}\newpage\fi\fi}
\makeatother
\makeatletter
\def\cleartorightpage{\clearpage\if@twoside \ifodd\c@page\else
\hbox{}\thispagestyle{toc}\newpage\fi\fi}
\makeatother
% Commencer les chapitres de l'dition en page impaire







%%%%%%%%%%%%%%%%%%%%%%%%%%%NOTES DE BAS DE PAGE ET APPARAT%%%%%%%%%%%%%%%%%%%%%%%%%%


%notes marginales
\usepackage{marginnote}
%notes marginales
\usepackage{multicol}
\maxdeadcycles=300%Résoud un problème de boucle: https://stackoverflow.com/questions/53842194/pdflatex-hang-after-large-number-of-figures
%%%%Notes de bas de page: appel non superscript, appel de note dans le text en exposant.
\makeatletter
\renewcommand\@makefntext[1]%
    {\noindent\makebox[0pt][r]{{\@thefnmark}\,. }#1}
\makeatother
%%%%Notes de bas de page: appel non superscript, appel de note dans le text en exposant.

%sets the footnote marker flush with,  but just inside the margin from, the text of the footnote; This option forces footnotes to the bottom of the page
\usepackage[bottom,flushmargin]{footmisc}
%sets the footnote marker flush with,  but just inside the margin from, the text of the footnote; This option forces footnotes to the bottom of the page

%%%%Indentation Parfaite Des Notes De Bas De Page
\makeatletter
\long\def\@makefntextFB#1{%
    \ifx\thefootnote\ftnISsymbol
        \@makefntextORI{#1}%
    \else
        \rule\z@\footnotesep
        \setbox\@tempboxa\hbox{\@thefnmark}%
            \ifdim\wd\@tempboxa>\z@
                \kern2em\llap{\@thefnmark.\kern0.5em}%
            \fi
        \hangindent2em\hangafter\@ne#1
    \fi}
\makeatother
%%%%Indentation Parfaite Des Notes De Bas De Page

%APPARAT

\usepackage[left,modulo]{lineno}%num\UTF{00E9}rotation des lignes

%sparation des numros de ligne
 \setlength\linenumbersep{0.3cm}
%sparation des numros de ligne

\usepackage{ednotes}%apparat
%run-in paragraph footnotes indented as ordinary footnotes, use the package with the para option:
\usepackage{manyfoot}
%run-in paragraph footnotes indented as ordinary footnotes, use the package with the para option:





%plusieurs niveaux de notes de bas de page
%\usepackage[ruled]{bigfoot}
\DeclareNewFootnote{B}[arabic]
\usepackage{perpage}
\MakePerPage{footnote}
\renewcommand{\thefootnote}{\alph{footnote}}
%\DeclareNewFootnote{A}[alph]
%
%plusieurs niveaux de notes de bas de page

% Dcaler le crochet fermant aprs le tmoin de la leon retenue (suppose de jouer aussi avec la feuille XSL)
 \newcommand{\Anotefmt}{% 
% % \renewcommand*{\sameline}[1]{\linesfmt{##1}}% 
% % \renewcommand*{\differentlines}[2]{\linesfmt{##1\textendash##2}}% 
% % \renewcommand*{\linesfmt}[1]{\textbf{##1}\enspace}% 
% % \renewcommand*{\pageandline}[2]{##1.##2}% ##1 page, ##2 line. 
% % \renewcommand*{\repeatref}[1]{##1}% E.g., ... 
% % % \renewcommand*{\repeatref}[1]{\textnormal{/}}% ... instead. 
% rgle d'origine
% % \renewcommand{\lemmafmt}[1]{##1\thinspace]\enskip}% 
% rgle d'origine
% nouvelle rgle
\renewcommand{\lemmafmt}[1]{##1~}% 
% % \renewcommand{\lemmaellipsis}{\textsymmdots}% 
% % \renewcommand{\notefmt}[1]{##1}% 
 } 
 % Dcaler le crochet fermant aprs le tmoin de la leon retenue (suppose de jouer aussi avec la feuille XSL)


%APPARAT



%%%%%%%%%%%%%%%%%%%%%%%%%%%%%%%%NOTES DE BAS DE PAGE%%%%%%%%%%%%%%%%%%%%%%%%%%%%

\newcommand{\hsp}{\hspace{20pt}}
\newcommand{\HRule}{\rule{\linewidth}{0.5mm}}

